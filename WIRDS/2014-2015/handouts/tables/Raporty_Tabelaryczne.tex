\documentclass[hyperref={pdfpagemode=FullScreen},table,green]{beamer}
\usepackage[polish]{babel}
\usepackage[Cp1250]{inputenc}
\usepackage[OT4]{fontenc}
\usepackage{beamerthemeshadow}
\usepackage{marvosym}
\usepackage{amsmath}
\usepackage{graphicx}
\usepackage{listings}
\usepackage{rotating}
\usepackage{fancyhdr}
\usepackage{lscape}
\usepackage{url}
\usepackage{xcolor}
\beamertemplateballitem
\beamertemplatenumberedballsectiontoc
\hypersetup{
  pdftitle={Pozna� 2015},
  pdfauthor={Jan Nowak}}
\title[Uniwersytet Ekonomiczny w Poznaniu]{\Large{Tabele w pakiecie ztable}}

\author[WIRDS]{Jan Nowak}

\institute{
  Analityka Gospodarcza\\Uniwersytet Ekonomiczny w~Poznaniu}
\date{04.05.2015}
\centering
\newcounter{minisection}
\newcommand{\minisectionframe}[1]{
\frame{
	\begin{center}
	#1
	\end{center}
}}

\newcommand{\smallhref}[1]{{\footnotesize\href{#1}{\url{#1}}}}

\begin{document}

\frame{\titlepage}
\part{}
\section{Plan prezentacji}
\subsection{Plan prezentacji}
\frame{
\setbeamercovered{transparent}
	\begin{enumerate}
		\tiny{
		\item Cel 1
		\item Cel 2
		\item Cel 3
		}
	\end{enumerate}
	}

\part{ztable}
\section{Tabele w ztable}
\subsection{Pierwsza tabela w ztable}
\frame{
\begin{block}{Pierwsza tabela w ztable}
\begin{table}[!hbtp]
\begin{center}
\begin{normalsize}
\color{black}
\begin{tabular}{rrlrrrr}
\hline
&klm&woj&wojewodztwo&trb&zut&d21\\ 
\hline
1&1.00&02&1.00&11.00&5.00&2.00\\ 
2&1.00&02&1.00&6.00&1.00&2.00\\ 
3&1.00&02&1.00&3.00&1.00&1.00\\ 
4&1.00&02&1.00&12.00&5.00&2.00\\ 
5&1.00&02&1.00&1.00&5.00&1.00\\ 
6&1.00&02&1.00&12.00&1.00&1.00\\ 
\hline
\end{tabular}
\end{normalsize}
\end{center}
\end{table}
\color{black}
\end{block}
}

\subsection{Druga tabela w ztable}
\frame{
\begin{block}{Druga tabela w ztable}
\begin{table}[!hbtp]
\begin{center}
\begin{normalsize}
\color{black}
\begin{tabular}{ccccccc}
\hline
&klm&woj&wojewodztwo&trb&zut&d21\\ 
\hline
1&1.00&02&1.00&11.00&5.00&2.00\\ 
2&1.00&02&1.00&6.00&1.00&2.00\\ 
3&1.00&02&1.00&3.00&1.00&1.00\\ 
4&1.00&02&1.00&12.00&5.00&2.00\\ 
5&1.00&02&1.00&1.00&5.00&1.00\\ 
6&1.00&02&1.00&12.00&1.00&1.00\\ 
\hline
\end{tabular}
\end{normalsize}
\end{center}
\end{table}
\color{black}
\end{block}
}

\subsection{Kolejne tabele w R i ztable}
\frame{
\begin{table}[!hbtp]
\begin{center}
\begin{normalsize}
\color{black}
\begin{tabular}{rrlcrrcrrc}
\hline
\cellcolor{white} &\multicolumn{2}{c}{Zmienna 1 i 2}&&\multicolumn{2}{c}{Zmienna 3 i 4}&&\multicolumn{2}{c}{Zmienna 5 i 6}\\ 
\cline{2-3}\cline{5-6}\cline{8-9}
&klm&woj&&wojewodztwo&trb&&zut&d21\\ 
\hline
1&1.00&02&&1.00&11.00&&5.00&2.00\\ 
2&1.00&02&&1.00&6.00&&1.00&2.00\\ 
3&1.00&02&&1.00&3.00&&1.00&1.00\\ 
4&1.00&02&&1.00&12.00&&5.00&2.00\\ 
5&1.00&02&&1.00&1.00&&5.00&1.00\\ 
6&1.00&02&&1.00&12.00&&1.00&1.00\\ 
\hline
\end{tabular}
\end{normalsize}
\end{center}
\end{table}
\color{black}
}

\subsection{Model regresji w R i ztable}
\frame{
\begin{table}[!hbtp]
\begin{center}
\begin{normalsize}
\color{black}
\begin{tabular}{rrrrr}
\hline
&Oszacowanie&B��d&t&Pr($>|t|$)\\ 
\hline
(Wyraz wolny)&610.8742&15.2413&40.08&0.0000\\ 
dochg&0.5838&0.0041&141.11&0.0000\\ 
los&52.1508&4.2540&12.26&0.0000\\ 
\hline
\multicolumn{5}{l}{\footnotesize{\begin{minipage}[c]{0.564705882352941\linewidth}Call: lm(formula = wydg $\sim$ dochg + los, data = gospodarstwa)\end{minipage}}}\\ 
\end{tabular}
\end{normalsize}
\end{center}
\end{table}
\color{black}
}

\subsection{Analiza wariancji w ztable}
\frame{
\begin{table}[!hbtp]
\begin{center}
\begin{normalsize}
\color{black}
\begin{tabular}{rrrrrr}
\multicolumn{6}{l}{\footnotesize{Analiza wariancji}}\\ 
\multicolumn{6}{l}{\footnotesize{Zmienna: klm}}\\ 
\hline
&Df&Sum Sq&Mean Sq&F value&$Pr(>F)$\\ 
\hline
wydg&1&2241.89&2241.89&702.68&0.0000\\ 
Residuals&32449&103527.76&3.19&&\\ 
\hline
\end{tabular}
\end{normalsize}
\end{center}
\end{table}
\color{black}
}

\subsection{Podsumowanie}
\frame{
\setbeamercovered{transparent}
	\begin{enumerate}
		\tiny{
		\item Wniosek 1
		\item Wniosek 2
		\item Wniosek 3
		}
	\end{enumerate}
	}

\section{Literatura}
\subsection{Literatura}
\frame{
	\setbeamercovered{transparent}
	\begin{block}{Literatura}
\tiny{
\begin{thebibliography}{9}
\beamertemplatebookbibitems
\bibitem[Cra]{Cra} Michael J. Crawley (2012), ,,\textit{The R Book}'', John Wiley \& Sons, Ltd., 2nd Edition.
\beamertemplatearticlebibitems
\bibitem[Keo]{Keo} Keon-Woong Moon (2015), ,,\textit{Package ztable}''.
\end{thebibliography}}
\end{block}
}

\section[]{}
\subsection{}
\frame{
Dzi�kuj� za uwag�!
}
\end{document}

